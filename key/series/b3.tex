\documentclass{article}
\usepackage{amsmath}

\begin{document}

\noindent
Does $\displaystyle \sum_{n=1}^\infty \frac{(-1)^n}{n^2}$
diverge, converge absolutely, or converge conditionally?

\begin{itemize}

\item {\bf Solution.} The series $\displaystyle \sum_{n=1}^\infty \frac{(-1)^n}{n^2}$ is alternating.

Let $a_n = \frac{(-1)^n}{n^2}$. Then $b_n = |a_n| = \frac1{n^2}$. The sequence $b_n$ is decreasing and 
\[ \lim_{n \rightarrow \infty} b_n = 0.\]

By the Alternating Series Test, the series
$\displaystyle \sum_{n=1}^\infty \frac{(-1)^n}{n^2}$
converges.

Does $\displaystyle \sum_{n=1}^\infty \frac{(-1)^n}{n^2}$ converge absolutely or conditionally?
We study $\sum |a_n|$, namely $\displaystyle \sum_{n=1}^\infty \left|\frac{(-1)^n}{n^2}\right|$, which is the series 
$\displaystyle \sum_{n=1}^\infty \frac{1}{n^2}$, which converges by the $p$-seriee test, so 
$\displaystyle \sum_{n=1}^\infty \frac{(-1)^n}{n^2}$ converges absolutely.

\item {\bf Solution.}
We study $\sum |a_n|$, namely $\displaystyle \sum_{n=1}^\infty \left|\frac{(-1)^n}{n^2}\right|$, which is the series 
$\displaystyle \sum_{n=1}^\infty \frac{1}{n^2}$, which converges by the $p$-series test, so by the Absolute Convergence Test,
$\displaystyle \sum_{n=1}^\infty \frac{(-1)^n}{n^2}$
converges.

Since $\sum |a_n|$ converges, $\sum a_n$ converges absolutely.

\end{itemize}

\end{document}
