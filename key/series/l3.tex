\documentclass{article}
\usepackage{amsmath}
\usepackage[margin=1.0in]{geometry}
\usepackage{xcolor}

\begin{document}

\noindent
Does $\displaystyle \sum_{n=1}^\infty \frac{1}{(n+1)^n}$
diverge, converge absolutely, or converge conditionally?

\subsection*{Solution 1}

\begin{align*}
L&=\lim_{n \to \infty} \sqrt[n]{|a_n|}\\
&= \lim_{n \to \infty} \sqrt[n]{\left| \frac{1}{(n+1)^n} \right|}\\
&= \lim_{n \to \infty} \frac{1}{n+1}\\
&= 0.
\end{align*}
Since $L < 1$, the series $\displaystyle \sum_{n=1}^\infty \frac{1}{(n+1)^n}$ converges absolutely by the Root Test.

\subsection*{Solution 2: much longer than Solution 1}

In an attempt to use the Ratio Test, you'd have to consider the limit
\begin{align*}
L&=\lim_{n \to \infty} \left|\frac{a_{n+1}}{a_n}\right|\\
&= \lim_{n \to \infty} \left| \frac{1}{(n+2)^{n+1}}\frac{(n+1)^n}{1}\right|\\
&= \lim_{n \to \infty} \frac{(n+1)^n}{(n+2)^{n+1}},
\end{align*}
but it looks like there's not any really nice limit here, and while L'Hopital's rule applies, it will be quite a mess. Actually, it is possible to evaluate this limit using some algebra and some limit laws:
\begin{align*}
&= \lim_{n \to \infty} \frac{(n+1)^n}{(n+2)^{n+1}}\\
&= \lim_{n \to \infty} \frac{(n+1)^n}{(n+2)^n(n+2)}\\
&= \lim_{n \to \infty} \left[\left(\frac{n+1}{n+2}\right)^n \cdot \frac1{n+2}\right].
\end{align*}
We consider the limit 
\[\lim_{n \to \infty} \left(\frac{n+1}{n+2}\right)^n\]
by defining $y = \left(\frac{n+1}{n+2}\right)^n$. So $\ln y = n \ln \frac{n+1}{n+2}$. So
\[ \ln y = \frac{\ln \frac{n+1}{n+2}}{\frac1n}\]
By applying a limit to both sides,
\[ \lim_{n \to \infty}\ln y = \lim_{n \to \infty} \frac{\ln \frac{n+1}{n+2}}{\frac1n}\]
and the limit on the right uses L'Hopital's rule. (Be sure to pay attention to the Chain Rule and the Quotient Rule.)
\begin{align*}
&=\lim_{n \to \infty} \frac{ \frac{n+2}{n+1} \cdot \frac{(n+2)\cdot 1 - (n+1) \cdot 1}{(n+2)^2}}{\frac{-1}{n^2}}\\
&=\lim_{n \to \infty} \frac{-n^2}{(n+1)(n+2)}\\
&=-1.
\end{align*}
after two more applications of L'Hopital's rule. Thus, 
\[ \lim_{n \to \infty} y = e^{-1} = \frac1e.\]
We were originally considering the limit:
\begin{align*}
&\lim_{n \to \infty} \left[\left(\frac{n+1}{n+2}\right)^n \cdot \frac1{n+2}\right]
&= \left[ \lim_{n \to \infty} \left(\frac{n+1}{n+2}\right)^n \right] \cdot \left[\lim_{n \to \infty} \frac1{n+2}\right]
&= \frac1e \cdot 0
&= 0.
\end{align*}
and we could treat the limit of the product as a product of the limits because both limits existed. Since $L < 1$, the series  $\displaystyle \sum_{n=1}^\infty \frac{1}{(n+1)^n}$ converges absolutely by the Ratio Test.

\end{document}%%%%%%%%%%%%%%%%%

\begin{align*}
L&=\lim_{n \to \infty} \sqrt[n]{|a_n|}\\
&= \lim_{n \to \infty} \sqrt[n]{\left| \right|}\\
\end{align*}

\begin{align*}
L&=\lim_{n \to \infty} \left|\frac{a_{n+1}}{a_n}\right|\\
&= \lim_{n \to \infty} \left| \right|\\
\end{align*}

\begin{align*}
\lim_{n \to \infty} a_n
&= \lim_{n \to \infty} \\
\end{align*}


Since $\sum |a_n| = \sum a_n$, the series $\displaystyle \sum_{n=1}^\infty AAAAAAAAAAAAAA$ converges absolutely.

Since $|r| < 1$, the series ...  converges by the Geometric Series Test.

Since $|r| \geq 1$, the series ...  diverges by the Geometric Series Test.

The function $f(x)=\frac{}{}$ is continuous, positive, and decreasing on $[1,\infty)$.

\subsection*{Solution}

