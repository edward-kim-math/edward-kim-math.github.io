\documentclass{article}
\usepackage{amsmath}
\usepackage[margin=1.0in]{geometry}
\usepackage{xcolor}

\begin{document}

\noindent
Does $\displaystyle \sum_{n=2}^\infty \frac{1}{n^2-1}$
diverge, converge absolutely, or converge conditionally?

\subsection*{Solution 1}

Note $2 \leq n^2$ for all positive integers $n \geq 2$. By adding $n^2$ to both sides,
\[ n^2 + 2 \leq 2n^2.\]
By subtracting $1$ from both sides,
\[ n^2 \leq 2n^2-2\]
Factoring the right side,
\[ n^2 \leq 2(n^2-1)\]
Dividing both sides by $n^2(n^2-1)$ we get
\[ \frac{1}{n^2-1} \leq \frac{2}{n^2}.\]
Since the series $\sum \frac{2}{n^2}=2\sum\frac{1}{n^2}$ converges by the $p$-test, the series $\displaystyle \sum_{n=2}^\infty \frac{1}{n^2-1}$ converges by the Direct Comparison Test.
Since all terms of the series are positive, the series $\displaystyle \sum_{n=2}^\infty \frac{1}{n^2-1}$ converges absolutely.

\subsection*{Solution 2}

The series $\sum \frac1{n^2}$ converges by the $p$-test. Let $a_n = \frac1{n^2-1}$ and $b_n=\frac1{n^2}$. Then
\begin{align*}
\lim_{n \to \infty} \frac{a_n}{b_n}
&= \lim_{n \to \infty} \frac{n^2}{n^2-1}\\
&= \lim_{n \to \infty} \frac{2n}{2n} \text{ by L'hopital}\\
&= \lim_{n \to \infty} \frac11\\
&= 1
\end{align*}
Since this limit is a finite, positive number, by the Limit Comparison Test, the series $\displaystyle \sum_{n=2}^\infty \frac{1}{n^2-1}$ converges.
Since all terms of the series are positive, the series $\displaystyle \sum_{n=2}^\infty \frac{1}{n^2-1}$ converges absolutely.

\subsection*{Solution 3}

By partial fraction decomposition, we get
\[ \frac1{n^2-1} = \frac{1/2}{n-1} - \frac{1/2}{n+1}.\]
So, the originally given series can be rewritten as
\[\sum_{n=2}^\infty \left(\frac{1/2}{n-1} - \frac{1/2}{n+1}\right)\]
After cancellations, we get the sequence of partial sums:
\[ s_n = \frac{1/2}{1} + \frac{1/2}{2} - \frac{1/2}{n} - \frac{1/2}{n+1}\]
Thus,
\begin{align*}
\lim_{n \to \infty} s_n
&= \lim_{n \to \infty} \left(\frac{1/2}{1} + \frac{1/2}{2} - \frac{1/2}{n} - \frac{1/2}{n+1}\right)\\
&= \frac{1/2}{1} + \frac{1/2}{2}\\
&= \frac12 + \frac14\\
&= \frac34.
\end{align*}
So, the series converges (by definition) to $\frac34$.

Since all terms of the series are positive, the series $\displaystyle \sum_{n=2}^\infty \frac{1}{n^2-1}$ converges absolutely.

\end{document}%%%%%%%%%%%%%%%%%

\begin{align*}
L&=\lim_{n \to \infty} \sqrt[n]{|a_n|}\\
&= \lim_{n \to \infty} \sqrt[n]{\left| \right|}\\
\end{align*}

\begin{align*}
L&=\lim_{n \to \infty} \left|\frac{a_{n+1}}{a_n}\right|\\
&= \lim_{n \to \infty} \left| \right|\\
\end{align*}

\begin{align*}
\lim_{n \to \infty} a_n
&= \lim_{n \to \infty} \\
\end{align*}

\begin{align*}
\lim_{n \to \infty} \frac{a_n}{b_n}
&= \lim_{n \to \infty} \\
\end{align*}


Since $\sum |a_n| = \sum a_n$, the series $\displaystyle \sum_{n=1}^\infty AAAAAAAAAAAAAA$ converges absolutely.

Since $|r| < 1$, the series ...  converges by the Geometric Series Test.

Since $|r| \geq 1$, the series ...  diverges by the Geometric Series Test.

The function $f(x)=\frac{}{}$ is continuous, positive, and decreasing on $[1,\infty)$.

\subsection*{Solution}

